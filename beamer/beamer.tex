\documentclass{beamer}

\usepackage[utf8]{inputenc}
\usepackage[russian]{babel}
\usepackage{commath}
\usepackage{graphicx}
\usepackage{multicol}

\usetheme{Berlin}
\usecolortheme{seahorse}

\setbeamertemplate{footline}[frame number]
\setbeamertemplate{caption}[numbered]

\setbeamerfont{caption}{size=\scriptsize}

\title
{Идентификация несанкционированного доступа сотрудников корпорации к информационным источникам методами искусственного интеллекта}

\author[Харитонов К.О.] % (optional, for multiple authors)
{
    Работу выполнил: К.О. Харитонов \inst{1}\\
    \small Научный руководитель: Л.В. Лабунец\inst{1}\\
    \small Консультант: С.А. Евсифеев\inst{2}
}
\institute[MSTU, MSU] % (optional)
{
  \inst{1}
  Системы обработки информации и управления, МГТУ им. Баумана
  \and
  \inst{2}
  Факультет вычислительной математики и кибернетики, МГУ
}
\date{25 июня 2019 г.}

\begin{document}
% 1
\maketitle

% 2
\begin{frame}{Содержание}
    \tableofcontents
\end{frame}

% 3
% Согласно отчёту Ассоциации «Объединение сертифицированных специалистов по расследованию хищений» (Association of Certified Fraud Examiners, ACFE), проактивное наблюдение за данными в компании используется в 37% компаний, что является одной из самых непопулярных средств защиты информации, в то же время их И - один из самых высоких показателей по сравнению с другими мерами защиты информации
% Проактивное наблюдение предполагает безостановочное наблюдение за работой автоматизированной системы. Несанкционированные действия регистрируются экспертами информационной безопасности, которые принимают окончательное решение о разрешении инцидента. Невозможно отследить все возможные инциденты силами экспертов без привлечения средств автоматизации этого процесса. Для решения этой задачи применяются DLP-системы различного рода. Большое распространение имеют DLP-системы, основанные на правилах, однако их недостатком является рост сложности при увеличении входных данных
% В данной работе подробно рассматриваются этапы предварительной обработки данных и применения алгоритма DBSCAN для кластеризации пользователей по признакам, характеризующим их поведение
\section{Введение}
\begin{frame}{Введение}
    \begin{itemize}
        \item Проактивное наблюдение предполагает безостановочное наблюдение за работой автоматизированной системы.
        \item Проактивное наблюдение за данными в компании используется лишь в 37\% компаний.
        \item Использование проактивного наблюдения снижает потери от внутреннего фрода на 52\%.
        \item Для решения этой задачи применяются DLP-системы различного рода.
    \end{itemize}
\end{frame}

\begin{frame}{Введение}
    \begin{figure}
        \centering
        \includegraphics[scale=0.35]{figures/Monitoring_Costs.png}
        \caption{Ущерб от различных видов угроз в корпорациях}
    \end{figure}
\end{frame}

% 4
\begin{frame}{Цель и задачи работы}
    \textbf{Цель работы}: 
    разработать методику кластеризации пользователей по манере их поведения в ходе работы с внешними устройствами.
    \newline
    \textbf{Задачи, решаемые в работе}:
    \begin{itemize}
        \item Рассмотреть применение методов искусственного интеллекта для достижения цели работы.
        \item Разработать алгоритм предварительной обработки данных о взаимодействии пользователей с внешними устройствами.
        \item Предложить способ кластеризации пользователей по исходным данным.
        \item Реализовать визуализацию кластеров в пространстве малой размерности.
    \end{itemize}
\end{frame}

% 5
\section{Критический анализ существующих подходов}
\begin{frame}{Защита данных от утечек}
    \begin{itemize}
        \item Проактивное наблюдение предполагает безостановочное наблюдение за работой автоматизированной системы. 
        \item Невозможно отследить все возможные инциденты силами экспертов без привлечения средств автоматизации этого процесса. 
        \item Большое распространение имеют DLP-системы, основанные на правилах, однако их недостатком является рост сложности при увеличении входных данных. 
    \end{itemize}
\end{frame}

% 6
\begin{frame}{Cпособы определения и предотвращения угроз}
    \begin{itemize}
        \item Системы, основанные на правилах и регулярных выражениях;
        \item Анализ снимков базы данных;
        \item Непосредственное сравнение файлов;
        \item Частичное сопоставление документов;
        \item Использование категориальных признаков;
        \item Методы статистического анализа.
    \end{itemize}
\end{frame}

% 7
%   Для облегчения работы эксперта в компаниях используют программные комплексы, позволяющие и облегчающие задачу обнаружения и разрешения инцидентов. Традиционно такие комплексы в той или иной форме представляют собой экспертные системы.
\begin{frame}{Экспертные системы в компьютерной безопасности}
    \begin{figure}
        \centering
        \includegraphics[scale=0.2]{figures/Interface.png}
        \caption{Интерфейс DLP-системы}
    \end{figure}
   % картинка
\end{frame}

% 8
% В качестве следующего логического шага в развитии систем противодействия внутренним угрозам в компании предложено использование методов интеллектуального анализа данных и машинного обучения с целью повышения эффективности работы системы и уменьшения затрат на привлечение экспертов в предметной области. Такой шаг позволит для автоматизации принятия решения использовать не только знания экспертов, но и данные, собранные в компании ранее и характеризующие поведение сотрудников и их паттерны поведения.
% В данной работе рассматривается алгоритм, использующий отличающийся подход: вместо использования правил и политик доступа при помощи методов машинного обучения и искусственного интеллекта находятся и анализируются данные о взаимодействии пользователей с внешними устройствами, позволяющие кластеризовать пользователей по паттернам, характерным их поведению.
\section{Разработка метода}
\begin{frame}{Описание подхода}
    Предложенный подход использует методы машинного обучения для работы с данными о взаимодействии с внешними устройствами:
    \begin{itemize}
        \item Данные агрегируются по пользователям;
        \item Производится нормализация данных при помощи преобразования Бокса-Кокса;
        \item Полученные данные кластеризуются с использованием алгоритма DBSCAN;
        \item Кластеры пользователей визуализируются при помощи метода главных компонент и линейного дискриминантного анализа.
    \end{itemize}
\end{frame}

% 9
% Работа производилась с логом, содержащим 40000 записей.
% Исходя из этого, было произведено преобразование данных таким образом, чтобы для каждого пользователя, исходя из данных о доменах и учётных записях, были поставлены в соответствие его действия с внешними устройствами. Анализ этих данных позволит решить задачу обнаружения несанкционированных действий пользователя.
\begin{frame}{Исходные данные}
    \begin{itemize}
        \item Работа производилась с файлом логгирования, содержащим данные о взаимодействии пользователей с внешними устройствами в корпорации;
        \item Данные содержат 40000 записей.
        \item Данные были агрегированы по пользователям.
    \end{itemize}

    Пример полей в исходном файле:
    \begin{multicols}{2}
        \begin{itemize}
            \item Тип устройства;
            \item Путь к файлу;
            \item Имя процесса, обращающегося к файлу;
            \item Размер файла;
            \item Событие аудита;
            \item Момент времени обращения;
            \item Имя компьютера, с которого производилось обращение к файлу;
            \item Идентификатор пользователя.
        \end{itemize}
    \end{multicols}
\end{frame}


% 10
%Для оценки возможной ковариации значений была построена матрица ковариации.
% Следует заметить, что матрица в данном случае содержит признаки, корреляция которых с другими либо близка к единице, либо к нулю. Это значит, что признаки коллинеарны, и их использование в модели не приведёт к какому-либо изменению результата. их следует устранить с целью ускорения работы алгоритма кластеризации
\begin{frame}{Предварительная обработка данных}
    \begin{column}[T]{5cm}
        \centering
        \begin{figure}
            \includegraphics[width=\textwidth]{figures/Covariations_crp.png}
            \caption{\scriptsize Исходная матрица ковариации}
            \label{fig:f_s_tag}
        \end{figure}
    \end{column}
    \begin{column}[T]{5cm}
        \centering
        \begin{figure}
            \includegraphics[width=\textwidth]{figures/Covariations_2_crp.png}
            \caption{\scriptsize Редуцированная матрица ковариации}
            \label{fig:f_s_tag}            
        \end{figure}
    \end{column}
\end{frame}


% Поскольку для данной задачи кластеров лишь два — исследование предполагает отделение тех пользователей, которые совершают несанкционированные действия, от тех, которые производят легитимную работу с внешними устройствами, а поиск ядер этих кластеров при помощи вычисления квадратичного отклонения для всех точек может не привести к ожидаемым результатам, был применён также алгоритм DBSCAN. Данный алгоритм группирует точки, если они тесно расположены, при этом помечая как выбросы точки, которые находятся в областях с малой плотностью. Такой характер работы является более подходящим в рамках данного исследования, учитывая, что пользователи, производящие несанкционированные действия, будут формировать скорее множество выбросов, чем отдельный кластер, по причине большого количества вариантов несанкционированных действий. 
\begin{frame}{Выбор алгоритма кластеризации}
    \begin{itemize}
        \item Для данной задачи требуется обнаружение двух кластеров: легитимных пользователей и нарушителей.
        \item Алгоритм k-средних находит ядра кластеров, что не позволяет сформировать кластер, содержащий выбросы.
        \item Алгоритм DBSCAN позволяет определять выбросы как объекты отдельного кластера.
    \end{itemize}
    
    \begin{figure}
        \includegraphics[scale=0.3]{figures/DBSCAN_Example.png}
        \caption{Визуализация работы алгоритма DBSCAN}
    \end{figure}
\end{frame}

% PCA: Для каждого k = 0,1,2,...n  среди всех k-мерных линейных многообразий в setR^n найти такое L_k in setR^n, что сумма квадратов уклонений x_i от L_k  минимальна: 
% $$\sum_{i=1}^{\m} dist^{2}(x_i,L_k) \rightarrow min$$
% Линейный дискриминантный анализ (LDA – Linear Discriminant Analysis) [1] заключается в выборе проекции пространства изображений на пространство признаков таким образом, чтобы было минимизировано внутриклассовое и максимизировано межклассовое расстояние в пространстве признаков.
\begin{frame}{Визуализация данных}
    Для визуализации кластеров используем метод главных компонент и линейный дискриминантный анализ. Эти методы позволяют выделить наиболее важную часть информации, сократив размерность информационного пространства.

    \begin{figure}
        \centering
        \includegraphics[scale=0.2]{figures/PCA_LDA.png}
        \caption{Уменьшение размерности данных при помощи метода главных компонент и линейного дискриминантного анализа}
    \end{figure}
\end{frame}


\section{Анализ результатов моделирования}
\begin{frame}{Нормализация данных}
    Преобразование Бокса-Кокса приводит данные к нормальному распределению, при этом позволяя регулировать степень преобразования при помощи параметра $\lambda$.
    \begin{figure}
        \includegraphics[scale=0.40]{figures/Box-Cox_Transform.png}
        \caption{Применение преобразования Бокса-Кокса к данным}
    \end{figure}
\end{frame}

% Анализ показывает, что большая часть пользователей была сгруппирована в крупные кластеры (Class 0, Class 1), а выделяющиеся пользователи либо являются частью малого кластера Class 2, либо рассматриваются как шум. Это наблюдение очень важно, так как применение кластеризации позволяет эксперту судить о принадлежности конкретного пользователя к определённому кластеру, а сами кластеры описывают модель поведения пользователя.
\begin{frame}{Визуализация кластеризованных данных}
    
    \begin{minipage}[b]{0.45\linewidth}
        \centering
        \begin{figure}
            \includegraphics[width=\textwidth]{figures/DBSCAN.png}
            \caption{\scriptsize Проекция данных 1}
            \label{fig:f_s_tag}
        \end{figure}
    \end{minipage}
    \begin{minipage}[b]{0.45\linewidth}
        \centering
        \begin{figure}
            \includegraphics[width=\textwidth]{figures/DBSCAN_2.png}
            \caption{\scriptsize Проекция данных 2}
            \label{fig:f_s_tag}            
        \end{figure}
    \end{minipage}
\end{frame}

\begin{frame}{Результаты кластеризации данных}

    \begin{table}
        \begin{tabular}{| c | c |}
            \hline
            Идентификатор класса & Число объектов \\
            \hline
            Class -1 & 8 \\
            \hline
            Class 0 & 221 \\
            \hline
            Class 1 & 430 \\
            \hline
            Class 2 & 23\\
            \hline
        \end{tabular}
        \caption{Результаты кластеризации исходных данных}
    \end{table}
    \begin{itemize}
        \item Большая часть пользователей была сгруппирована в крупные кластеры (Class 0, Class 1);
        \item Выделяющиеся пользователи либо являются частью малого кластера Class 2, либо рассматриваются как шум.
    \end{itemize}{}
\end{frame}

% lda, pca картинки
\begin{frame}{Уменьшение размерности}
    Сократив размерность до трёх измерений, мы сможем расположить кластеры в пространстве меньшей размерности.
    \begin{minipage}[b]{0.45\linewidth}
        \centering
        \begin{figure}
            \includegraphics[width=\textwidth]{figures/PCA.png}
            \caption{\scriptsize Метод главных компонент}
            \label{fig:f_s_tag}
        \end{figure}
    \end{minipage}
    \begin{minipage}[b]{0.45\linewidth}
        \centering
        \begin{figure}
            \includegraphics[width=\textwidth]{figures/LDA.png}
            \caption{\tiny Линейный дискриминантный анализ}
            \label{fig:f_s_tag}            
        \end{figure}
    \end{minipage}
\end{frame}

% 15
\section{Заключение}
\begin{frame}{Заключение}
    \textbf{В данной работе}:
    % В рамках данной выпускной квалификационной работы был разработан алгоритм определения несанкционированного доступа пользователей информационной системы к данным на основе методов машинного обучения. В качестве метода нормализации данных было рассмотрено преобразование Бокса-Кокса. Для кластеризации данных был использован плотностной алгоритм пространственной кластеризации с присутствием шума. Наконец, для уменьшения размерности данных были применены метод главных компонент и линейный дискриминантный анализ.
    % В качестве итогового результата приводится визуализация принадлежности пользователя к кластеру, которая в дальнейшем может быть использована для оценки поведения пользователя в контексте взаимодействия с данными. При анализе результатов кластеризации было обнаружено разделение пользователей на крупные кластеры пользователей с похожим поведением и малые, определяющие необычное их поведение, что объясняется спецификой исходных данных. Полученные при помощи статистических и графических методов показатели могут быть применены для улучшения и развития систем DLP. 
    \begin{itemize}
        \item Рассмотрена проблема предотвращения несанкционированного доступа к файлам в корпорациях;
        \item Были изучены подходы, используемые на данный момент в системах анализа утечек данных, и представлено их дальнейшее развитие;
        \item Был разработан алгоритм определения несанкционированного доступа пользователей информационной системы к данным на основе методов машинного обучения;
        \item Реализация описанного алгоритма применена в реальной задаче анализа поведения пользователей.
    \end{itemize}
\end{frame}

\end{document}
